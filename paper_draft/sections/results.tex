\section{Results}\label{sec:results}
\para{Main metrics.} \Tabref{tab:main_results} reports overlap, cosine, causal patching, and probe metrics. Top-$k$ overlap between compound and constituent contexts is low (Jaccard 0.11--0.14), and bootstrap means remain between 0.09 and 0.11, indicating limited shared sparse features. In contrast, the compositionality probe reconstructs compound embeddings with mean cosine 0.996 and MSE 5.19, a large improvement over the \wbaseline baseline (MSE 12.49).

\begin{table}[t]
\centering
\caption{Main results for feature overlap, patching, and compositionality.
Best values per metric are in \textbf{bold}.}
\resizebox{0.95\textwidth}{!}{%
\begin{tabular}{@{}lcl@{}}
\toprule
Metric & Value & Notes \\
\midrule
Compound--washing Jaccard & 0.136 & Top-$k$ overlap ($k=50$) \\
Compound--machine Jaccard & 0.111 & Top-$k$ overlap ($k=50$) \\
Compound--union Jaccard & 0.129 & Top-$k$ overlap ($k=50$) \\
Compound unique fraction & 0.68 & Unique features in top-$k$ \\
Bootstrap mean (compound--washing) & 0.092 [0.064, 0.136] & 200 samples \\
Bootstrap mean (compound--machine) & 0.105 [0.087, 0.124] & 200 samples \\
Bootstrap mean (compound--union) & 0.114 [0.092, 0.137] & 200 samples \\
Cosine(compound, washing) & 0.578 & Mean latent vectors \\
Cosine(compound, machine) & 0.041 & Mean latent vectors \\
Causal patching $\Delta$logit & -0.019 $\pm$ 0.109 & $n=5$, $p=0.75$ \\
Probe MSE (ridge) & \textbf{5.19} & Lower is better \\
Probe MSE (\wbaseline baseline) & 12.49 & Lower is better \\
Probe mean cosine (ridge) & \textbf{0.996} & Higher is better \\
\bottomrule
\end{tabular}%
}
\label{tab:main_results}
\end{table}


\para{SAE overlap and patching plots.} \Figref{fig:sae_overlap} visualizes overlap distributions, while \Figref{fig:causal_patching} summarizes patching outcomes. Patching compound residuals into ``washing process'' contexts yields a mean $\Delta$logit of $-0.019$ with high variance ($\pm 0.109$) across five template pairs.

\begin{figure}[t]
    \centering
    \includegraphics[width=0.95\linewidth]{figures/sae_overlap.png}
    \caption{SAE top-$k$ overlap between compound and constituent contexts. Overlap remains low across comparisons, consistent with weak shared sparse features.}
    \label{fig:sae_overlap}
\end{figure}

\begin{figure}[t]
    \centering
    \includegraphics[width=0.95\linewidth]{figures/causal_patching.png}
    \caption{Causal patching effects on the ``machine'' logit. The mean effect is near zero with high variance, indicating no reliable causal lift from the patched compound activations at layer 6.}
    \label{fig:causal_patching}
\end{figure}

\para{Statistical context.} The patching result is not significant ($p=0.75$) and uses $n=5$ template pairs, so we interpret it as weak evidence against a single dominant compound feature at this layer.
