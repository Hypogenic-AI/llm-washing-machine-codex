\section{Results}
\label{sec:results}

\para{SAE feature overlap is low.} \Tabref{tab:main_metrics} shows that compound--constituent top-50 Jaccard overlaps are 0.11--0.14, and 68\% of compound features are unique relative to constituent top-50 sets. \Figref{fig:sae_overlap} visualizes the overlap patterns, reinforcing the weak feature sharing signal.

\para{Compositionality probe is strong.} The ridge probe predicts compound embeddings from constituents with cosine 0.996 and MSE 5.19, outperforming the head-noun (w2) baseline (MSE 12.49), a 58.4\% reduction. This indicates that compound representations are largely reconstructible from constituents despite low \sae overlap.

\para{Causal patching shows weak effects.} Patching \compound activations into \washing process prompts at layer 6 does not increase the logit for ``machine'' (mean $\Delta$logit $-0.019 \pm 0.109$, $n=5$). \Figref{fig:causal_patching} shows the distribution of patching effects and their variability across templates.

\begin{table}[t]
\centering
\caption{Main results for feature overlap, patching, and compositionality.
Best values per metric are in \textbf{bold}.}
\resizebox{0.95\textwidth}{!}{%
\begin{tabular}{@{}lcl@{}}
\toprule
Metric & Value & Notes \\
\midrule
Compound--washing Jaccard & 0.136 & Top-$k$ overlap ($k=50$) \\
Compound--machine Jaccard & 0.111 & Top-$k$ overlap ($k=50$) \\
Compound--union Jaccard & 0.129 & Top-$k$ overlap ($k=50$) \\
Compound unique fraction & 0.68 & Unique features in top-$k$ \\
Bootstrap mean (compound--washing) & 0.092 [0.064, 0.136] & 200 samples \\
Bootstrap mean (compound--machine) & 0.105 [0.087, 0.124] & 200 samples \\
Bootstrap mean (compound--union) & 0.114 [0.092, 0.137] & 200 samples \\
Cosine(compound, washing) & 0.578 & Mean latent vectors \\
Cosine(compound, machine) & 0.041 & Mean latent vectors \\
Causal patching $\Delta$logit & -0.019 $\pm$ 0.109 & $n=5$, $p=0.75$ \\
Probe MSE (ridge) & \textbf{5.19} & Lower is better \\
Probe MSE (\wbaseline baseline) & 12.49 & Lower is better \\
Probe mean cosine (ridge) & \textbf{0.996} & Higher is better \\
\bottomrule
\end{tabular}%
}
\label{tab:main_results}
\end{table}


\begin{figure}[t]
    \centering
    \includegraphics[width=0.95\linewidth]{figures/sae_overlap.png}
    \caption{Top-$k$ \sae feature overlap between compound and constituent contexts. Overlaps are low, and compound-specific features dominate the top-50 set.}
    \label{fig:sae_overlap}
\end{figure}

\begin{figure}[t]
    \centering
    \includegraphics[width=0.95\linewidth]{figures/causal_patching.png}
    \caption{Causal patching effects at layer 6. Patching compound activations into \washing process contexts yields no consistent increase in the ``machine'' logit and shows substantial variance across $n=5$ templates.}
    \label{fig:causal_patching}
\end{figure}
