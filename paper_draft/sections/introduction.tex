\section{Introduction}\label{sec:introduction}
\para{Compounds stress concept localization.} Mechanistic interpretability and model editing often assume that a concept corresponds to a distinct direction or feature in the residual stream. That assumption is plausible for simple entities but less obvious for compounds like ``washing machine,'' where meaning may arise from composition rather than a single localized feature.

\para{Why this matters.} If compound concepts are distributed across constituent features, single-direction edits can be incomplete or misleading. This question affects how we interpret circuits, how we localize knowledge, and how we intervene on models to change behavior.

\para{What is missing?} Prior work has shown superposition and polysemanticity in internal representations \cite{elhage2022superposition} and has introduced sparse feature discovery through \sae dictionary learning \cite{cunningham2023sparse,anthropic2023monosemanticity}. Causal tracing and editing methods localize behavior to specific components \cite{wang2022ioi,meng2022rome}, while neuron-level attribution identifies important units \cite{dai2021knowledgeneurons}. However, direct tests for concrete compound nouns that combine sparse features, causal patching, and compositionality probes are limited.

\para{Our approach.} We examine \gpttwo with a pretrained \sae at layer 6 and mine \wikitext contexts containing ``washing machine,'' ``washing,'' or ``machine.'' We compare sparse feature overlap, test causal effects with activation patching, and train a ridge probe that predicts the compound embedding from constituents (see \figref{fig:method_overview}).

\para{Quantitative preview.} We find low top-$k$ feature overlap (Jaccard 0.11--0.14; bootstrap means 0.09--0.11), no reliable causal lift from patching (mean $\Delta$logit $-0.019$), and near-perfect compositional predictability (probe cosine 0.996; MSE 5.19 vs 12.49 baseline).

In summary, our main contributions are:
\begin{itemize}[leftmargin=*,itemsep=0pt,topsep=0pt]
    \item We propose a concrete compound-noun testbed that aligns sparse features, causal patching, and probing in a single analysis.
    \item We conduct \sae overlap and causal patching experiments showing weak evidence for a single compound-specific feature at layer 6.
    \item We demonstrate strong compositional structure via a ridge probe that reconstructs compound embeddings from constituents with cosine 0.996.
    \item We document limitations and implications for concept-level interventions.
\end{itemize}

\para{Organization.} \secref{sec:related_work} reviews prior work. \secref{sec:methodology} details the data, models, and metrics. \secref{sec:results} reports results and figures. \secref{sec:discussion} discusses implications and limitations.
