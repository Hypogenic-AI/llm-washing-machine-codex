\section{Discussion}
\label{sec:discussion}

\para{Interpretation.} The low \sae overlap suggests that compound contexts activate a distinct set of latents, but the compositional probe indicates that compound embeddings are almost perfectly predictable from constituents. Taken together, these results support a view where compound meaning is compositional in representation geometry even if \sae features appear unique at a single layer.

\para{Limitations.} The compound contexts are synthetic because \wikitext contains no literal \compound examples, which limits ecological validity. The analysis targets a single model and a single layer with one pretrained \sae, and the causal patching uses only $n=5$ template pairs. Finally, \sae features are not canonical units, so uniqueness at the feature level does not imply atomicity of meaning.

\para{Implications.} For concept editing and steering, these findings argue against assuming that compound nouns correspond to single directions. Interventions should consider multi-feature and multi-layer compositions, and should be evaluated with both geometric and causal diagnostics.

\para{Broader impacts.} Interpretability claims about concept locality can influence safety decisions and downstream edits. Overstating atomicity risks brittle interventions; emphasizing compositionality encourages more conservative and robust control strategies.
